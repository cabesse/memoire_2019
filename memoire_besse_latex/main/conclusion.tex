\part*{Conclusion}
\addcontentsline{toc}{part}{Conclusion}
\markboth{Conclusion}{Conclusion} 

Étudier la numérisation de masse, tout en participant à la construction d'une telle initiative a permis de nourrir un intéressant dialogue entre problématiques du terrain et théorie. Les enjeux de la numérisation de masse se dessinant au fil de notre état de l'art et de la création de la feuille de route, l'apport pratique a contribué à élargir le champ initial de la recherche, et la recherche a pu influencer la réalité des propositions. La réalisation de notre mission de stagiaire au sein du projet Time Machine, notre collaboration à l'élaboration des propositions de la feuille de route et la rédaction du présent mémoire, nous ont demandé un important travail de réflexion sur trois axes : la définition des projets de numérisation (contexte, histoire, caractéristiques et enjeux), l'analyse de précédentes initiatives (motivations et réponses apportées aux enjeux identifiés), l'étude du projet Time Machine (réponses apportées aux enjeux, innovations, risques et opportunités, potentiel impact). Nous proposons d'esquisser un bilan des propositions contenues dans ces trois parties de notre mémoire ainsi que des enseignements tirés de notre expérience de stagiaire.

La première partie, entièrement théorique, nous a permis d'abord de retracer l'origine des premières initiatives de numérisation, mêlant développements technologiques et élargissement des activités des institutions culturelles et patrimoniales. Les développements numériques tendant à s'accélérer, les années 1990 voient apparaître les précurseurs des projets de numérisation de masse et les années 2000 marquent la naissance des premières recommandations et directives européennes, appelées à favoriser la croissance de l'économie numérique et à contribuer à l'orientation de ces initiatives d'envergure, la photogrammétrie figurant désormais en bonne place dans les objectifs de numérisation européens. Time Machine s'inscrit ainsi dans la continuité des premiers projets de bibliothèque universelle qui se voyaient également limités par des développements technologiques, dont la taille nécessitait un besoin de standardisation et ne manquaient pas de voir des décisions politiques ou historiques influencer la création de leurs collections. L'histoire nous montre que l'avènement du numérique a très vite impliqué une grande mixité parmi les acteurs des projets, qui se sont confrontés aux questions du droit d'auteur et ont \oe{}uvré au déploiement de solutions politiques.

Dans un deuxième temps, nous nous sommes attelée au \textit{comment} des initiatives de numérisation de masse, à la définition de leurs caractéristiques et aux enjeux reliés. Par nature complexes, ces initiatives ne se laissent pas facilement résumer et semblent gagner en difficultés à mesure que l'on cherche à les comprendre. Nous nous sommes basée sur notre analyse historique et expérience de stagiaire afin de proposer une sélection de cinq enjeux à étudier plus précisément (la collaboration, le financement public-privé, le droit d'auteur, l'interopérabilité, le stockage sur le long-terme). 

Au vu de la taille des projets, appelés à articuler intérêts divers et pratiques multiples, la collaboration apparaît comme un facteur clé de réussite. Une numérisation massive, implique des coûts conséquents dont les institutions publiques ne peuvent, seules, prendre la responsabilité. Des partenariats public-privé semblent inévitables, mais rendent plus difficiles l'inscription de ces projets dans une démarche d'ouverture et de transparence. Les directives actuelles du droit d'auteur ne facilitent pas la circulation des données numériques et préviennent la représentation de celles issues de notre passé le plus récent, Time Machine devra oser la prise de risque s'il ne veut pas être limité par ces barrières. Sortir des \inquote{silos} peut se faire à condition de veiller à une interopérabilité technique et des politiques documentaires à la hauteur des ambitions des projets de numérisation de masse, en arrière-fonds la collaboration semble encore détenir les clés du succès. Appréhender la préservation des données pour en favoriser les usages est essentiel à tout projet de numérisation de masse. Time Machine aura pour tâche de comprendre les nuances propres à chaque enjeu et d'intégrer la matérialité des solutions au sein de son infrastructure. La création de cette infrastructure ne se fera pas sans la mise en place d'un cadre opérationnel à même de réconcilier les pratiques du monde de la documentation, du web et de l'industrie, afin de planifier et exécuter la transformation des données du passé en données numériques.

Notre travail nous permet de conclure que si le projet veut réussir à établir de nouvelles normes et apporter des réponses satisfaisantes à ces différents enjeux, il lui faudra non seulement disposer de grands moyens financiers, mais être capable d'instaurer un dialogue privilégié avec chaque membre de son réseau, tout en composant avec le cadre externe du projet (légal et politique) qui, sans avoir apporté de réponses complètes aux problématiques du droit d'auteur, nourrit de hautes ambitions pour la numérisation du patrimoine européen. Alors que le financement du projet est des plus incertain, Time Machine doit réussir à se positionner comme réseau d'influence, à même d'accroître le soutien politique dont il bénéficie et espérer ainsi briser certaines barrières territoriales et légales qui rendent compliquée la réalisation du \gls{bigd} du passé.

La deuxième partie est consacrée à l'étude des motivations d'autres projets de numérisation de masse (\textit{Google Books}, Europeana, \textit{HathiTrust} et \textit{the Digital Public Library of America}) et aux réponses apportées par ces initiatives aux enjeux de la numérisation. Cette comparaison nous a permis de déduire que les frontières séparant ces projets sont poreuses, chacun se basant sur les réalisations des autres pour construire de nouveaux développements. Les initiatives sont conscientes de cette complémentarité et travaillent ensemble à élaborer de nouvelles solutions aux différents enjeux. Nous concluons cette recherche par le constat que plutôt que de chercher à catégoriser ces initiatives, il semble plus intéressant de les considérer comme un tout, afin de pouvoir déplacer le spectre de la réflexion au niveau de leur ensemble et ne pas seulement s'arrêter sur les détails qui les composent. Time Machine devra lui aussi développer des ponts avec ces grands réseaux, et trouver un équilibre entre le phénomène de mondialisation culturelle qui les accompagne et la préservation des particularités locales et régionales de ses partenaires.

Enfin notre troisième partie, plus concrète, est d'abord pour nous l'occasion de présenter les propositions contenues dans la feuille de route en analysant les réponses apportées aux différents enjeux par Time Machine. Puisque l'agenda, entre les recherches conduites pour le mémoire et les échéances de rendu européennes, n'a pas toujours coïncidé, nous proposons un certain nombre d'amendements. Ils visent à accroître la collaboration avec les \gls{agr}s que sont les Time Machines locales, à mieux prendre en compte les usages du public dans la gestion des droits d'auteur, à favoriser la préservation sur le long-terme par des directives techniques et à compléter les outils destinés à garantir l'interopérabilité afin de sortir de la logique des \inquote{silos}. Nous argumentons ensuite que ce qui distingue Time Machine des autres entreprises de numérisation de masse, est l'intégration de toutes les parties publiques ou privées ou sein d'un même réseau (le projet ne faisant pas la distinction entre collaborateurs internes et futurs exploitants), et le déploiement d'innovations technologiques, pour certaines basées sur l'intelligence artificielle, au service de la création du \gls{bigd} du passé. Fort de ces propositions novatrices, Time Machine accroît sa popularité auprès d'acteurs attirés par les nouvelles perspectives offertes par la démarche inclusive et la crédibilité technologique du projet.

Notre travail d'analyse historique et des enjeux, nous a permis de définir un certain nombre d'éléments constituants des risques ou opportunités. Time Machine devra, au-delà du déploiement d'infrastructures et routines cohérents, proposer une plateforme en accord avec les besoins et attentes des différentes communautés composant son public-cible en évitant les méthodologies limitées au monde occidental, pour ne pas biaiser le développement de ses collections et cloisonner les usages futurs. Favoriser la cocréation devrait permettre à Time Machine de proposer des solutions fédératrices au-delà des frontières territoriales et académiques. Résolument tourné vers le futur, le projet se devra également de refléter les nouvelles pratiques induites par le numérique et proposer au grand public un rôle actif et créateur de valeur, en promouvant les \gls{cs}. Time Machine a l'opportunité de pouvoir donner à l'Europe une plateforme aux contenus égaux et éthiques, ne cherchant pas à refléter l'histoire telle que déjà écrite, mais à en offrir une nouvelle reproduction. 

Nous concluons cette partie sur le constat que si de nombreuses études se consacrent aux enjeux techniques et organisationnels liés aux projets de numérisation de masse, trop peu de chercheurs s'intéressent aux questions du \textit{pourquoi}. Pourtant la portée de ces projets dépasse les frontières du monde culturel et patrimonial et ces réseaux détiennent une forme de pouvoir susceptible d'avoir des répercussions sur notre société. 

Réseau définitivement tourné vers les innovations numériques, Time Machine semble déjà incarner les attentes d'une classe dirigeante européenne inquiète de bénéficier des retombées économiques offertes par ce nouveau monde. S'inscrivant dans le suivi des politiques de mondialisation, et s'adressant à une audience plus large que celle réservée aux institutions culturelles et patrimoniales, le projet ouvre le dialogue au sein des plus hautes sphères du pouvoir européen et suscite avant même le déploiement de son infrastructure, l'intérêt des foules. Ce nouvel acteur des initiatives de numérisation de masse semble déjà détenir une forme de pouvoir. Seules des recherches plus avancées permettront de définir avec assurance si ce dernier est limité aux partenaires de son réseau, où s'étend au-delà et touche la sphère politique. Nous pensons que pour permettre l'étude de ces nouveaux phénomènes et être en adéquation avec leur complexité, il est plus que temps de considérer les acteurs de la numérisations de masse comme de nouvelles entités informationnelles, dont les contours évoluent certes, mais entraînent des répercussions en dehors du milieu des institutions culturelles et patrimoniales, qui viennent ébranler un certain ordre établi.

A l'instar des précédentes initiatives de numérisation de masse, le projet Time Machine semble apporter autant de nouvelles solutions que de questionnements. Si le projet aboutit et réussit une intégration équilibrée de tous ces éléments, il deviendra sans doute un acteur informationnel, politique, économique et culturel de poids. Espérons alors, que la prophétie du \textit{mirrorworld} se trompe lorsqu'elle décrit les futurs services développés comme appelés à devenir des biens payants au même titre que l'eau et l'électricité\footcite{kelly_ar_2019}, et que Time Machine demeurera un outil au service de l'égalité et de la démocratie, libre et gratuit.