\part*{Introduction}
\addcontentsline{toc}{part}{Introduction}
\markboth{Introduction}{Introduction}

Les projets de numérisation de masse semblent issus des conséquences de la révolution numérique, ils sont portés en Europe par le constat effrayant d'un retard face aux entreprises et à la croissance américaines.

Pour rappel, en 2006, le monde comptait 50 sociétés de taille mondiale, dont 17 étaient européennes. Dix ans plus tard, 31 sont américaines et 8 sont chinoises. En 2017 Nestlé, le géant de l'alimentaire demeure seul représentant de l'industrie européenne\footcite{battu_histoire_2018}. 

Bénéficiant de soutiens politiques, les projets de numérisation s'inscrivent également dans l'histoire du développement des industries culturelles et patrimoniales et semblent incarner l'évolution logique des bibliothèques numériques. Perçus par les uns comme une réponse à la faible vélocité du marché économique européen, et un moyen incontournable permettant de garantir l'accessibilité au savoir dans une société aux pratiques de plus en plus connectées pour les autres\footcite{wu_building_2011}, que sont réellement ces projets de numérisation de masse et que nous apprennent-ils sur l'évolution de notre société ? 

Symptôme de la révolution numérique, l'information, la connaissance et le savoir créent de la valeur économique et sont de plus en plus traités comme \inquote{un bien immatériel appropriable}\footcite[p.105]{mattelart_histoire_2018}. Ce constat voit s'affronter logiques du droit d'auteur et des brevets, accusés de privatiser le monde, et tentatives de définition des \inquote{biens publics communs} visant à préserver l'accessibilité d'un certain savoir à tous. 

Le concept de patrimoine culturel européen date seulement des années 1950 et n'a été ancré dans l'agenda européen qu'à partir des années 1990. Présenté par la Commission, à la fois comme vecteur de l'identité culturelle et européenne, le patrimoine culturel est un élément important de la mémoire de l'Europe\footcite{thelle_persuasive_2011}. Or dans un contexte de mondialisation, au-delà de la question de la préservation du savoir, se pose le besoin de préserver cette diversité culturelle. Comment protéger dans cette course à la grandeur numérique, les spécificités intrinsèques au bien-être humain ? L'idée d'un universalisme culturel est-elle seulement envisageable\footcite{laulan_diversite_2018} ?

Alors que nous sommes sur le point de disposer des technologies rendant possible la réalisation du rêve de la \inquote{Cité mondiale} de Paul Otlet\footcite{thylstrup_politics_2018}, les acteurs actuels du marché de la numérisation de masse suscitent craintes, critiques et admiration. 

Le projet Time Machine, pour lequel nous avons effectué notre stage du 15 avril au 31 août 2019 au sein du \gls{dhlab} de l'\gls{epfl}, s'inscrit dans la continuité de ces débats idéologiques, ses objectifs trouvant une troublante incarnation dans l'idée récemment décrite du \textit{mirrorworld}\footnote{Si le terme bénéficie d'une nouvelle popularité, il a été utilisé pour la première fois par l'informaticien de Yale, David Gelernter en 1991, dans son livre \inquote{\textit{Mirror Worlds: Or the Day
Software Puts the Universe in a Shoebox...How It Will Happen and What It Will Mean}}(Oxford University Press, 1991)}  : 

\begin{quotation}
[\textit{Traduction}]
Nous sommes à l'aube de la création d'une nouvelle plateforme, qui numérisera le monde. Sur cette plateforme, lieux et choses seront lisibles par les machines, sujets au pouvoir des algorithmes. Quiconque dominera cette grande plateforme s'inscrira parmi les plus riches et puissantes puissances de l'histoire. [...] L'histoire deviendra un verbe. D'un geste de la main, vous pourrez voyager dans le temps [...] ou dans le futur. Ces différents scénarios auront le goût de la réalité, car ils seront dérivés d'une reproduction à l'échelle de notre monde actuel. En ce sens, peut-être faut-il plus parler d'un monde en quatre dimensions que d'un miroir.\footnote{\inquote{We are now at the dawn of the third platform, which will digitize the rest of the world. On this platform, all things and places will be machine-readable, subject to the power of algorithms. Whoever dominates this grand third platform will become among the wealthiest and most powerful people and companies in history [...]. History will be a verb. With a swipe of your hand, you will be able to go back in time, at any location, and see what came before. [...] Or you'll scroll in the other direction: forward. [...] These scroll-forward scenario will have the heft of reality because they will be derived from a full-scale present world. In this way, the mirrorworld may be best referred to as a 4D world.}\cite{kelly_ar_2019}}.
\end{quotation}

Sans prétendre apporter une réponse à tous les enjeux techniques, culturels, politiques, légaux que soulèvent les entreprises de numérisation de masse, nous 
proposons dans ce mémoire quelques clés pour mieux en comprendre les origines et la complexité afin d'interroger leur positionnement face aux autres acteurs de l'information. Car si la vision du \textit{mirrorworld} semble résumer les motivations d'un projet tel que Time Machine, de nombreux débats et évolutions doivent encore être menés, qui influenceront fortement la future forme de ce double numérique et viendront bouleverser l'organisation de notre monde réel.

Notre mémoire est structuré en trois parties, visant à la fois à rendre compte du travail réalisé durant notre stage et à apporter un cadre théorique et réflexif pour comprendre le contexte dans lequel nous avons été amenée à élaborer certaines solutions. Dans la première partie, nous vous proposons de découvrir l'histoire des projets de numérisation de masse, ou comment les premières initiatives ont évolué vers des projets de grandes envergures. Ces derniers sont issus de révolutions technologiques, mais découlent également de l'évolution des pratiques bibliothéconomiques induites par ces mêmes révolutions. Leurs grandes portées les inscrit de plus dans une histoire économique et par conséquent politique, dont nous vous résumerons les étapes. Dans un deuxième temps, nous présenterons les caractéristiques de ces initiatives et les différents enjeux \textit{(amener différents acteurs à collaborer, financement et partenariats public-privé, droit d'auteur, sortir des silos ou la quête de l'interopérabilité, stockage sur le long-terme et préservation)} qui en découlent. Ceci afin de mieux appréhender les questions soulevées par le \textit{comment} de la numérisation.

Notre projet de recherche étant motivé par notre expérience de stagiaire, nous poserons le contexte et la description du projet Time Machine dans la dernière section de cette première partie, abordant le contexte institutionnel, l'histoire de la recherche en humanités numériques et le lien entre ce secteur académique et les projets de numérisation. Les précédents développements menés par le laboratoire en vue de l'élaboration de Time Machine seront également rappelés (dont notamment une présentation de \textit{Venice Time Machine}). Time Machine étant à l'heure actuelle un consortium réunissant quelque centaines d'institutions culturelles et patrimoniales sous la coordination du laboratoire lausannois, nous présenterons plus en détails l'organisation et les objectifs de cette initiative. 

Le deuxième temps de notre mémoire, offre un regard plus précis sur quatre initiatives de numérisation de masse emblématiques du 21\up{e} siècle, \textit{Google Books}. Europeana, \textit{HathiTrust} et \textit{\gls{dpla}}. Nous proposerons une brève analyse des différentes réponses apportées par ces initiatives aux enjeux de la numérisation préalablement identifiés et conclurons sur la question des divergences et similitudes entre ces différents projets.

Nous consacrerons la troisième partie à notre expérience de stagiaire, dévouée à l'élaboration d'une feuille de route pour les opérations et l'infrastructure de Time Machine sur une échelle de dix ans, et sur les propositions contenues dans ladite feuille de route, qui nous ont permis de nous confronter à chacun des enjeux présentés. Nous présenterons notre travail et les réponses apportées aux questions soulevées par ces enjeux dans un premier temps. Puis nous analyserons les différentes innovations apportées par Time Machine par rapport aux précédentes initiatives, et les risques et opportunités auxquels il devra faire face. Nous conclurons par un élargissement sur les motivations et impacts soulevés par l'envergure d'un tel projet : le \textit{pourquoi} des entreprises de numérisation de masse, et nous tenterons de définir si Time Machine s'inscrit dès lors dans la continuité des entreprises de numérisation portées par les acteurs culturels et patrimoniaux, ou s'établit en tant que nouvel acteur de l'information.








