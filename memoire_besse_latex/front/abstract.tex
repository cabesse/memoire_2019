\chapter*{Résumé}
\addcontentsline{toc}{chapter}{Résumé}
\markboth{Résumé}{} 
Ce mémoire a été réalisé en vue de l'obtention du diplôme de Master 2 \inquote{Technologies numériques appliquées à l'histoire } de l'École nationale des chartes. Il retrace le travail que nous avons mené lors d'un stage de quatre mois réalisé au sein du Laboratoire d'humanités digitales de l'École polytechnique fédérale de Lausanne, au bénéfice du projet Time Machine. Cette initiative européenne, portée par un consortium d'institutions et de réseaux culturels et patrimoniaux, sous la coordination du laboratoire lausannois, vise entre autres à organiser la numérisation à l'échelle européenne et proposer de nouveaux paradigmes d'accès aux données numérisées par le biais des technologies d'intelligence artificielle, ajoutant par exemple un facteur temporel aux plateformes traditionnelles. Ce mémoire porte sur l'élaboration d'une feuille de route pour l'infrastructure et les opérations du projet, planifiée afin d'atteindre une phase de maturité sous dix ans. Pour mieux appréhender la complexité du travail effectué, il s'agit également d'un rapport sur l'historique, les politiques culturelles, les caractéristiques et les enjeux qui découlent des initiatives de numérisation de masse. Une analyse plus critique exposera les réponses apportées par Time Machine aux différents enjeux de la numérisation, les écueils qu'il reste à éviter et le potentiel impact d'un projet de cette envergure sur le monde politique culturel et patrimonial européen, posant ainsi la question de la place occupée par ces initiatives au sein de cet environnement d'acteurs.

\medskip

%informations à compléter

\textbf{Mots-clefs:} accessibilité ; bibliothèques numériques ; droit d'auteur ;  Europeana ; \textit{HathiTrust} ; humanités numériques ; \textit{Google Books} ; interopérabilité ;  intelligence artificielle ;  moteur de recherche diachronique ; numérisation de masse ; partenariats public-privé ; politiques culturelles européennes ; surveillance de masse ; \textit{the Public Library of America} ; valorisation \\

% informations à compléter
\textbf{Informations bibliographiques:} Camille Besse, \textit{Numérisation de masse : vers la création d'un nouvel acteur de l'information - le projet Time Machine}, mémoire de master \og Technologies numériques appliquées à l'histoire \fg{}, Thibault Clérice et Frédéric Kaplan, École nationale des chartes, 2019.

\clearpage
\thispagestyle{empty}
\cleardoublepage