\chapter*{Remerciements}
\addcontentsline{toc}{chapter}{Remerciements}
\markboth{Remerciements}{} 


Mes remerciements s'adressent d'abord à mon maître de stage, le Professeur Frédéric Kaplan, détenteur de la chaire d'humanités digitales de l'EPFL, pour les échanges vivifiants qui ont rythmé les quatre mois de mon expérience, et surtout pour la confiance qu'il m'a accordée dès les premiers jours. Je remercie les membres du projet Time Machine, François Ballaud pour m'avoir guidée durant mes premières semaines à travers les complexités du projet, ainsi qu'Isabella di Lenardo pour avoir partagé son expertise acquise sur le projet vénitien. Un immense merci à Kevin Baumer, coordinateur de Time Machine, pour m'avoir fait découvrir les coulisses du projet, et partagé mes questionnements de stagiaire. J'ai grandement apprécié son écoute bienveillante et sa constante bonne humeur.
Je remercie également chaleureusement mes collègues et autres stagiaires du DHLAB, pour m'avoir accueillie généreusement au sein de leur équipe, m'avoir aidée à comprendre les processus technologiques si délicats de Time Machine et apporté leur soutien indéfectible durant la durée de mon stage. C'est grâce à eux que j'ai pu garder le cap face à la complexité de ma tâche. Je remercie  les bibliothécaires du \textit{Rolex Learning Center}, pour m'avoir épaulée face aux subtilités de la gestion des données de la recherche au sein d'un projet d'ampleur européenne, ces moments de partage me furent très précieux. 

Je remercie vivement mes professeurs Gautier Poupeau (\textit{data architect} à l'INA) et Clément Oury (Adjoint au chef du service Conservation et Numérisation, Bibliothèque du Muséum d'Histoire naturelle), pour m'avoir prodigué conseils et avis et pris le temps de me recevoir, lorsque j'en ai eu besoin.

Je tiens encore à adresser mes remerciements à Thibault Clérice, directeur du master et tuteur de stage pour m'avoir permis d'intégrer le master et aidée à prendre du recul par rapport à mes activités de stagiaire. 

Enfin un merci spécial à ma relectrice de toujours, Danièle Besse, qui n'est jamais effrayée par l'ampleur de la tâche, et à mon compagnon Dimitri Wyss pour m'avoir soutenue durant ces deux années parisiennes.


