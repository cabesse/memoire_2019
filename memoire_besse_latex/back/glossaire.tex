\newglossaryentry{agr}{ 
name={agrégateur}, 
description={Dans le cadre de ce mémoire, ce terme désigne une organisation offrant un service de collecte, validation, harmonisation, stockage et souvent enrichissement des données issues de projets de numérisation des collections d'institutions culturelles et patrimoniales\dots}}

\newglossaryentry{appauto}{ 
name={apprentissage automatique ou machine learning}, 
description={Champ d'étude de l'intelligence artificielle qui se fonde sur des approches statistiques pour donner aux ordinateurs la capacité \inquote{d'apprendre} à partir de jeux de données. Cet apprentissage peut se faire de manière supervisée (le modèle mathématique utilise des informations sur les jeux de données entrant et sur les résultats attendus), semi-supervisée (certains jeux de données entrant ne sont pas explicités dans le modèle mathématique et l'ordinateur doit déduire leurs fonctions), ou non supervisée (le modèle mathématique contient uniquement des informations sur les données entrantes et aucune sur les transformations attendues en résultat)\footnote{\cite{noauthor_apprentissage_2019}}\dots}}

\newglossaryentry{appr}{ 
name={apprentissage profond ou deep learning}, 
description={L'apprentissage profond fait partie de la famille des méthodes d'apprentissage automatique ou machine learning. Basé sur les réseaux de neurones, cette méthode utilise un système de couches pour extraire progressivement des informations\footnote{\cite{noauthor_deep_2019}}\dots}}

\newglossaryentry{db}{ 
name={bases de données}, 
description={Une base de données permet de stocker des données brutes ou informations en rapport avec un thème ou une activité. Ces dernières permettent ensuite de rechercher les informations ainsi stockées. Il existe différentes natures de base de données\footnote{\cite{noauthor_base_2019}}\dots}}

\newglossaryentry{bigd}{
name={big data}, 
description={Grands jeux de données, dont l'analyse par traitements informatiques permet de mettre en valeur d'insoupçonnés motifs et associations, contribuant à enrichir notre connaissance sur le genre humain\footnote{\cite{noauthor_big_2019}}\dots}}

\newglossaryentry{cc}{
name={Creative Commons}, 
description={Association à but non lucratif dont l'objectif est d'offrir une solution alternative légale (par le biais de plusieurs licences) aux personnes souhaitant libérer leurs oeuvres des droits de propriété intellectuelle standards dans leurs pays, lorsque jugés trop restrictifs\footnote{\cite{noauthor_creative_2019}}\dots}}

\newglossaryentry{cur}{
name={curation}, 
description={Néologisme servant à désigner la pratique visant à sélectionner, éditer et partager des ressources pertinentes du web en réponse à une requête donnée.\footnote{\cite{noauthor_curation_2019}}\dots}}

\newglossaryentry{graph}{
name={graphe}, 
description={Dans le contexte de ce mémoire, ce terme est utilisé pour désigner une grande base de connaissance, compilant les informations de plusieurs sources différentes\footnote{\cite{noauthor_knowledge_2017}}\dots}}

\newglossaryentry{ia}{
name={intelligence artificielle}, 
description={Ensemble des théories et techniques mises en oeuvre en vue de réaliser des machines capables de simuler l'intelligence\footnote{\cite{noauthor_intelligence_2019}}\dots}}

\newglossaryentry{oa}{ 
name={libre accès ou Open Access}, 
description={Mise à disposition de manière pérenne, libre, gratuite et en ligne, des travaux de recherches financés par les pouvoirs publics\footnote{\cite{noauthor_libre_2019}}\dots}}

\newglossaryentry{moteur}{ 
name={moteur d'inférence ou \textit{Inference Engine}}, 
description={Dans le domaine de l'\gls{ia}, les moteurs d'inférence sont les composants d'un système appliquant des règles logiques sur une base de données afin d'en déduire de nouvelles informations. La base de donnée contient des faits avérés, le moteur applique des règles logiques à cette base et en déduit ce nouveau savoir (Y est né telle année, X a rencontré Y = X est probablement né entre telles années)\footnote{\cite{noauthor_inference_2019}}\dots}}

\newglossaryentry{nua}{ 
name={nuages de points}, 
description={Ensemble de points de données dans un système de coordonnées à trois dimensions, format des données issues d'une numérisation 3D. Dans le cadre d'un processus de numérisation, il faudra souvent rassembler plusieurs nuages de points afin d'obtenir une vue complète du bâtiment\footnote{\cite{noauthor_nuage_2018}}\dots}}

\newglossaryentry{num}{ 
name={numérisation}, 
description={Dans le contexte du projet Time Machine, ce terme indique non seulement la conversion des informations d'un support (texte, image, audio, vidéo, artefact) ou signal électrique en données numériques\footnote{\cite{noauthor_numerisation_nodate}}, mais également la transformation du patrimoine bâti, élément géographique, en modèles 3D\dots}}

\newglossaryentry{os}{ 
name={Open Science}, 
description={Mouvement visant à rendre les données de la recherche au sens large (jeux de données, notes de laboratoire, processus, images etc.), accessibles à tous\footnote{\cite{noauthor_open_2019}}\dots}}



\newglossaryentry{photo}{ 
name={photogrammétrie}, 
description={La photogrammétrie peut se résumer en une technique de reconstruction numérique en 3D d'un objet physique, permettant de déterminer les dimensions et les volumes des objets à partir de mesures effectuées sur des photographies montrant les perspectives de ces objets\footnote{\cite{noauthor_photogrammetrie_2019}}\dots}}

\newglossaryentry{poly}{ 
name={POLY-perspective}, 
description={Concept indiquant que les futurs chercheurs et scientifiques devraient adopter une approche plurielle pour mener à bien leurs activités, et ne pas hésiter à collaborer avec des spécialistes d'autres domaines pour proposer des solutions innovantes\footnote{\cite{noauthor_cdhs_nodate}}\dots}}

\newglossaryentry{ree}{
name={reconnaissance d'entités nommées}, 
description={La reconnaissance d'entités nommées est une sous-tâche de l'activité d'extraction d'informations dans des corpus documentaires, consistant à rechercher des objets textuels (mots ou groupes de mots) catégorisables dans des classes (noms de personnes, de lieux, quantités, distances, valeurs, dates etc.)\footnote{\cite{noauthor_reconnaissance_2018}}\dots}}

\newglossaryentry{reo}{
name={reconnaissance optique de caractères}, 
description={La reconnaissance optique de caractères, ou océrisation, désigne les procédés informatiques pour la traduction d'images, de textes imprimés ou dactylographiés en fichiers de texte\footnote{\cite{noauthor_reconnaissance_2017}}\dots}}

\newglossaryentry{reneu}{
name={réseaux de neurones artificiels}, 
description={Système imitant le fonctionnement des neurones biologiques, fait partie des technologies d'\gls{appr}, utilisé également par l'\gls{ia}. Ce réseau implique le traitement d'une information en couches successives, permettant d'affiner les résultats proposés par la dernière couche\footnote{\cite{noauthor_artificial_2019}}\dots}}

\newglossaryentry{cs}{
name={sciences citoyennes ou \textit{citizen science}}, 
description={Désigne la recherche scientifique conduite entièrement ou en partie par des scientifiques amateurs ou non-professionnels. Ce mouvement est également décrit comme la participation du public dans la recherche scientifique\footnote{\cite{noauthor_citizen_2019}}\dots}}


\newglossaryentry{tal}{
name={traitement automatique du langage naturel}, 
description={Domaine multidisciplinaire impliquant la linguistique, l'informatique et l'\gls{ia} visant à créer des outils de traitement de la langue naturelle pour diverses applications\footnote{\cite{noauthor_traitement_2019}}\dots}}

\newacronym{api}{API}{\textit{Application Programming Interface}}
\newacronym{ascii}{ASCII}{\textit{American Standard for Information Interchange}}
\newacronym{bnf}{BNF}{\textit{Bibliothèque Nationale de France}}
\newacronym{csa}{CSA}{\textit{Coordination and Support Actions}}
\newacronym{dh}{DH}{\textit{Digital Humanities}, humanités digitales, humanités numériques}
\newacronym{dhlab}{DHLAB}{\textit{Digital Humanities Laboratory} ou Laboratoire d'humanités digitales}
\newacronym{dpla}{DPLA}{\textit{the Digital Public Library of America}}
\newacronym{epfl}{EPFL}{École Polytechnique fédérale de Lausanne}
\newacronym{glam}{GLAM}{\textit{Galleries, Libraries, Archives, Museums} ou Galeries, Bibliothèques, Archives, Musées}
\newacronym{http}{HTTP}{HyperText Transfer Protocol}
\newacronym{iaa}{IA}{Intelligence Artificielle}
\newacronym{iiif}{IIIF}{International Image Interoperability Framework}
\newacronym{oai}{OAI-PMH}{Open Archives Initiative Protocol for Metadata Harvesting}
\newacronym{rdf}{RDF}{Resource Description Framework}
\newacronym{rfc}{RFC}{Request for Comments}
\newacronym{ue}{UE}{Union Européenne}
\newacronym{unesco}{UNESCO}{\textit{The United Nations Educational, Scientific and Cultural Organization} ou Organisation des Nations Unies pour l'éducation, la science et la culture}
\newacronym{uri}{URI}{Uniform Resource Identifier}
\newacronym{w3c}{W3C}{\textit{World Wide Web Consortium}}
\newacronym{wp}{WP}{\textit{Work Package}, lot de travail}


